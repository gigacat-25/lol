\documentclass[a4paper,12pt]{article}
\usepackage{amsthm}
\newtheorem{theorem}{Theorem}[section]
\newtheorem{definition}[theorem]{Definition}
\newtheorem{corollary}[theorem]{Corollary}
\newtheorem{lemma}[theorem]{Lemma}
\begin{document}
\section{Theorem}
\begin{theorem}
Let A be the hypotenuse B be the base and C be the height the theorem states that the theorem states that in a right angle triangle the square of hypotenuse is sum of square of other two sides
\end{theorem}
\section{Definition}
\begin{definition}
The pythagoras theorem states that in a right angle triangle the square of hypotenuse is sum of square of other two sides
\end{definition}
\section{Corollary}
\begin{corollary}

\[a^2 = b^2 + c^2\] ,right angle triangle
\[a^2 < b^2 + c^2\] ,acute
\[a^2 > b^2 + c^2\] ,obtuse
\end{corollary}
\section{lemma}
\begin{lemma}
Triangle which has same base and height of a square has the same area as the half area of the square
\end{lemma}
\end{document}